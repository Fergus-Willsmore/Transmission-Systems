\documentclass{article}

% This Document is designed to house the writing for the 
% concepts learnt for my MPhil.
% Created by F.R. Willsmore

% Packages
\usepackage[utf8]{inputenc}
\usepackage[top=1in, bottom=1.25in, left=1.25in, right=1.25in]{geometry}
\usepackage{graphicx} 
\usepackage{fancyhdr}
\usepackage{amsmath}
\usepackage{amssymb}
\usepackage{lipsum}
\usepackage{bm}
\usepackage{array}
\usepackage{multirow}
\usepackage{caption}
\usepackage[dvipsnames]{xcolor}
\usepackage{parskip}
\usepackage[toc,page]{appendix}
\usepackage[hidelinks]{hyperref}
\usepackage{float}

% Commands
\DeclareMathOperator*{\argmin}{arg\,min}
\DeclareMathOperator*{\argmax}{arg\,max}
\newcommand{\ie}{\textit{i.e. }}
\newcommand{\eg}{\textit{e.g. }}
\setlength\parindent{24pt}

% struts for table environments
\newcommand\Tstrut{\rule{0pt}{2.6ex}}         % = `top' strut
\newcommand\Bstrut{\rule[-0.9ex]{0pt}{0pt}}   % = `bottom' strut
\newcommand{\exedout}{%
  \rule{0.8\textwidth}{0.5\textwidth}%
}

\hypersetup{
    colorlinks,
%    linkcolor={red!50!black},
%    citecolor={blue!50!black},
%    urlcolor={blue!80!black}
    linkcolor={blue},
    citecolor={red},
    urlcolor={blue}
}
\renewcommand*{\sectionautorefname}{Section} % capitalises Section in \autoref

% Headers
\pagestyle{fancy}
\lhead{MPhil}
\chead{}
\rhead{Modelling Cascade Failure}
\lfoot{}
\cfoot{\thepage}
\rfoot{}

% Title
\title{Concepts Learnt for MPhil}
\author{Fergus Willsmore}
\date{\today}

\begin{document}
\maketitle

\newpage

\tableofcontents

\newpage

\section{Glossary}

\begin{itemize}
\item[] {\it interconnection system} conveys power from generators to demand locations (loads). 
\item[] {\it transmission system} carries power at high voltages, typically over long distances. 
\item[] {\it distribution system} distributes power within a local geographical area at low voltage. 
\item[] {\it buses} represent the nodes of the network and may house generators and others represent where a distribution system (a load) is attached.
\item[] {\it lines} represent the power lines used to transmit power between buses.
\item[] {\it transformers} are used to convert between different voltages.
\item[] {\it real} power $P$ is the power actually supplied to the load.
\item[] {\it reactive} power $Q$ is the power that bounces back and fourth between the load and generator (energy lost due to electric and magnetic fields).
\item[] {\it complex} power $S$ is the effective power produced by alternating current, $S=P+iQ$.
\item[] {\it resistance} $r$ is the force against the flow of current offered by the material of the conductor.
\item[] {\it reactance} $x$ is the resistance offered by inductors and capacitors with respect to time.
\item[] {\it impedance} $z$ is the effective resistance to alternating current, $z=r+ix$. 
\item[] {\it conductance} $g$
\item[] {\it susceptance} $b$
\item[] {\it admittance} $y=g+ib$

\end{itemize}


\section{Power Grid}

Electrical flow can be thought of as the movement of electrons through a surface, referred to as the {\it current} ($I$). The movement of charge is provoked by the {\it voltage} ($V$) of an electrical circuit. The potential difference creates an electric field that acts on the electrons within the circuit. This is best thought of as electrical potential energy, such that as voltage is increased more force is required to hold the electrons in a fixed position. Once released the electron will accelerate toward the positive pole resulting in the conversion of electrical potential energy into heat and kinetic energy. The idea of {\it electric power}, combines the elements {\it current} and {\it voltage} such that 
\begin{equation}
P=V\times I
\end{equation}
expressed in watts. There are two distinct types of power flows, namely {\it direct current} (DC) and {\it alternating current} (AC). The simpler of the two is DC power, where the electrons always flow in the same direction (such as battery operated circuits). In a battery operated circuit the electrons always move from the negative pole to the positive pole. However in an AC circuit, the direction of current periodically changes, which gives rise to complex behaviour. 


\subsection{AC vs DC}

Electrical {\it current} is the flow of electrons through a surface measured as a rate. There are two different types; alternating current and direct current. The main distinction between the two is that the flow of electrons for DC is linear, \ie, the electrons (- charge) will move directly to the positive charge, whereas in AC the direction of flow alternates direction periodically with fixed frequency. The benefits of DC is that it is more efficient at electrical transmission over short distances due to the linear flow and limited power loss. However, the benefit of AC is that the voltage can be easily changed by a transformer thus allowing high voltage transmission for larger distances. Therefore the majority power is sourced from AC generators. 

\section{Bienstock 2016}
\subsection{Chapter 4}

There is some disagreement in the initial point of cascade. The initial line failure should not trigger a cascade as the system should be N-1 safe, therefore still in a controlled position. However if co susceptible lines also fail then the cumulative outages begin to grow slowly. At some point, the cumulative loss of power lines is so rapid that no control is possible. The initial point of cascade for our purposes is from the initial line failure as this leads to the uncontrollable state.\\

A common theme in the survey of large-scale blackouts, seemed to be that the initial line failure had a large capacity (such as 345kV). The redistribution of power flows seemed to put stress on other high capacity lines causing sag and heat leading to them also tripping. The Italian blackout was a bit different, in that Italy is a big importer of power. This means that the generation of power within Italy is very limited (similar to SA situation). A loss of an importing line caused a cascade which quickly lead to ``syncronism loss" (frequency instability) causing the island to blackout.\\

Prior modelling techniques have outlined a few features of cascade failure such as line tripping due to overloads is primarily used, focus on slow cascade processes (ideally start slow then fast), DC approximation to power flows and ignore short-term dynamics (such as islanding). \\

Cascade model 1, aims to capture the idea of upgrading equipment after failure. In the case of lines, this would be increasing their rating (capacity). This models this long term cascade behaviour, but the short term cascade model is based on non-decreasing probability function. In the first instance that a line becomes overloaded
\begin{equation}
P(\text{line } j \ \text{becomes outaged}) = h^0(|f_{j,t}|/U_{j,t}^{\text{max}})=c
\end{equation}
for all lines $j$. Then after redistribution overloaded lines outage with probability $h^1>c$. Another quantity pointed out is average relative line load on a given day $t$,
\begin{equation}
\bar{M}_t \dot{=}\frac{\sum_j M_{jt}}{m}
\end{equation}
where $m$ indicates the number of lines. It was found that for a given $\mu$ the systems seems to tend toward a long term ``dynamic equilibrium". This suggests that for a constant rate of demand growth line upgrades is not sufficient to avert blackouts.\\

Island dynamics appear to require complex modelling techniques in terms of the physical dynamics of electricity flow and operating decisions. The main factor deciding the fate of the island is the deficit between the demand and generation. Small deficits can be managed by increasing generation up until capacity, whereas larger deficits will result in island wide blackout due to severe overload (protective mechanism). The initial model presented is a linear program to model economical redispatch of the generators with high penalty on load shedding. Alternatively, Pfitzner (2011) recognises that redispatch optimisation problems in islanding is not efficient enough for operating decisions. New outputs for generators in the island are set due to the ratio of demand to generation. In the event that demand is too high causing overload in generators,load is shed by intentionally tripping lines. The aim of which is to reduce the overall size of the cascade. The tripping rule (A1) trips the overloaded line with minimum flow achieved the best results.  




\end{document}

\bibliographystyle{acm}
\bibliography{../mphil_lib.bib}